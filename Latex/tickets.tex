
\section{Caso 98: Analizar Documento: Descripción de Flujos y Procesos de la Plataforma de Correo de CANTV.NET }

\begin{description}

\item[Asunto] Analizar Documento: Descripción de Flujos y Procesos de la Plataforma de Correo de CANTV.NET\item[Descripción] El documento se encuentra en el repositorio cantv-info.git bajo la carpeta
cantv.net, de igual manera se adjunta el documento a este caso.\item[Propietario] cmaldonado\item[Horas Trabajadas] 8

\item[Comentarios] Lista de comentarios agregados a este caso:  
\begin{enumerate}
        \item {\bfseries Walter Vargas (walter@covetel.com.ve)  } - {\bfseries 2011-07-17 19:14:12} \\ Se debe comparar el diagrama anexo 1 del presente documento con los diagramas
de la página dos y cuatro del documento PDF Visio-Flujos.Correos.pdf        \item {\bfseries Walter Vargas (walter@covetel.com.ve)  } - {\bfseries 2011-07-17 20:01:01} \\ El diagrama de la página 4 del documento Visio-Flujos.Correos.pdf es
exactamente igual al diagrama de la página 21 del documento
INST-ODA-OAI-042-flujos-de-correo.pdf

Se adjuntan las imágenes de la comparación.        \item {\bfseries Walter Vargas (walter@covetel.com.ve)  } - {\bfseries 2011-07-17 20:09:38} \\ En nuestro caso, nos importan los flujos que afecten al correo para el dominio
cantv.com.ve.

El flujo F1 afecta todo el correo entrante generado por MTAs o MTUs a dominios
cuyo puntero MX corresponda a RELAY.CANTV.NET

En este caso, el puntero MX del dominio cantv.com.ve corresponde a
RELAY.CANTV.NET como se ve en la siguiente prueba con el comando dig

$ dig @200.44.32.12 cantv.com.ve MX ; <<>> DiG 9.6.0-APPLE-P2 <<>>
@200.44.32.12 cantv.com.ve MX ; (1 server found) ;; global options: +cmd ;; Got
answer: ;; ->>HEADER<<- opcode: QUERY, status: NOERROR, id: 13132 ;; flags: qr
rd ra; QUERY: 1, ANSWER: 1, AUTHORITY: 0, ADDITIONAL: 0 ;; QUESTION SECTION:
;cantv.com.ve. IN MX ;; ANSWER SECTION: cantv.com.ve. 3600 IN MX 10
relay.cantv.net. ;; Query time: 76 msec ;; SERVER:
200.44.32.12#53(200.44.32.12) ;; WHEN: Sun Jul 17 15:38:05 2011 ;; MSG SIZE
rcvd: 61        \item {\bfseries Walter Vargas (walter@covetel.com.ve)  } - {\bfseries 2011-07-17 20:32:15} \\ Hemos intentado registrar una cuenta de correo en CANTV.NET para hacer algunas
pruebas, pero esto no ha sido posible, debido a que el formulario de registro
de cuenta nueva [1] no esta funcionando.

[1]
http://registro.cantv.net/main/inicio.asp?CATALOG_PLAN=FREEMAIL_R_BF&imagen=correo_e_gratuitol.gif        \item {\bfseries Walter Vargas (walter@covetel.com.ve)  } - {\bfseries 2011-07-17 20:56:31} \\ Para realizar una pequeña prueba de correo, actualizamos los datos y
solicitamos la contraseña nueva para la cuenta

walter_varg1@cantv.net que corresponde a uno de los servicios de ABA que
Covetel posee.

El proceso tomo unos 10 minutos, luego de eso, se escribió un correo desde la
cuenta walter_varg1@cantv.net a la cuenta waltervargas@gmail.com. Este correo
fue clasificado como SPAM por los servidores de Google.

Es importante determinar la razón por la cual esta cuenta de correo esta
marcada como SPAM por parte de Google.        \item {\bfseries Walter Vargas (walter@covetel.com.ve)  } - {\bfseries 2011-07-17 22:41:39} \\ El documento indica en la sección C.1 "Flujo de mensajes de la CAPA SMTP" en el
flujo F06 la siguiente descripción entre otras:

i. Envío en formato SMS a RA.MOVILNET.COM.VE, de notificación de recepción de
correos en cuantas del dominio CANTV.NET que cuentan con el servicio TUN-TUN
(de correos recibidos por la capa RELAY.CANTV.NET).

Nos preguntamos si este documento esta realmente actualizado, pues el servicio
TUN-TUN de CANTV.NET actualmente esta fuera de servicio, de hecho esto se
comprueba entrando a la página web www.tun-tun.com y podrá observar un texto
que dice "En estos momentos este servicio está inactivo".

Se adjunta el screenshot de la página www.tun-tun.com        \item {\bfseries Walter Vargas (walter@covetel.com.ve)  } - {\bfseries 2011-07-17 23:02:55} \\ Se asume que el acrónimo NDR en el párrafo ii. de la sección C.1 del documento
hace referencia al acrónimo NDR que significa: Non Delivery Report/Recelpt,
correo electrónico automático que envía el sistema de correo informando a quien
envía el correo acerca de un problema en la entrega del correo.

Referencias:

http://en.wikipedia.org/wiki/Non_delivery_report        \item {\bfseries Walter Vargas (walter@covetel.com.ve)  } - {\bfseries 2011-07-17 23:27:31} \\ Se debe preguntar a los funcionarios de CANTV.NET de que se trata o que
significa DNS-D.

Nuestro equipo supone que es un servicio que permite consultarle vía el
protocolo DNS los atributos "homedirectory" y "quota" de una cuenta de correo.

DNS-D interactúa con la capa POSTMAN y la base de datos de el BOSS de CANTV
según el anexo 1 de este documento.        \item {\bfseries Carlos Maldonado (cmaldonado@covetel.com.ve)  } - {\bfseries 2011-08-11 18:29:38} \\ documento culminado, segunda versión pendiente con imágenes actualizadas sobre el flujo de datos en la plataforma de correo, ya que han cambiado ciertos aspectos    \end{enumerate}

\end{description}

\section{Caso 93: Analizar Documento Diseño y Operación Solución OpenLDAP }

\begin{description}

\item[Asunto] Analizar Documento Diseño y Operación Solución OpenLDAP\item[Descripción] Se adjunta el documento para su análisis, también se encuentra en el
repositorio cantv-info en la carpeta cantv.net\item[Propietario] walter\item[Horas Trabajadas] 42

\item[Comentarios] Lista de comentarios agregados a este caso:  
\begin{enumerate}
        \item {\bfseries Walter Vargas (walter@covetel.com.ve)  } - {\bfseries 2011-07-18 22:42:19} \\ En la página 4 del presente documento se observa un ejemplo del esquema que
esta utilizando el servidor OpenLDAP de CANTV.NET para representar una entrada
en el directorio. Están utilizando el siguiente esquema para representar el
atributo DN de la entrada: dn:
mail=rmartinezp@cantv.net,ou=hash_qrQR,ou=correo,ou=isp,dc=cantv,dc=net Este
esquema para construir el atributo DN no es estándar debido a que no utiliza
los atributos sugeridos en la tabla de la sección 3 en el RFC 4514 descrita a
continuación: String X.500 AttributeType ------
-------------------------------------------- CN commonName (2.5.4.3) L
localityName (2.5.4.7) ST stateOrProvinceName (2.5.4.8) O organizationName
(2.5.4.10) OU organizationalUnitName (2.5.4.11) C countryName (2.5.4.6) STREET
streetAddress (2.5.4.9) DC domainComponent (0.9.2342.19200300.100.1.25) UID
userId (0.9.2342.19200300.100.1.1) Se adjunta el RFC citado.        \item {\bfseries Walter Vargas (walter@covetel.com.ve)  } - {\bfseries 2011-07-19 12:16:12} \\ El documento indica que OpenLDAP no soporta más de 32 base de datos.

Esta información no fue encontrada en el manual oficial de OpenLDAP, se procede 
a realizar una prueba técnica para confirmar esta información. 

        \item {\bfseries Walter Vargas (walter@covetel.com.ve)  } - {\bfseries 2011-07-19 13:36:09} \\ El servicio DNS-D ya no es utilizado por la capa POSTMAN, 
ahora las consultas de los atributos homedirectory y quota 
son vía LDAP hacia la capa "ldap.ric2.cantv.net" 

Esta información se encuentra en el correo envíado a Roberto de Oliveira por
parte de Reynaldo R. Martinez en fecha 02 Jun 2011 en donde dice textualmente:

 2.- La capa postman no hace las consultas ya vía DNS-D (de hecho no
 existe esa capa de dns-d para correo). Las consultas son ahora vía LDAP
 hacia la capa "ldap.ric2.cantv.net".        \item {\bfseries Walter Vargas (walter@covetel.com.ve)  } - {\bfseries 2011-07-19 13:47:30} \\ Saludos Roberto,  

La prueba que se esta haciendo es crear más de 32 bases de datos para OpenLDAP
en prueba y levantar el servicio para confirmar este límite que indica el
documento de CANTV.NET . 

De igual manera seria bueno consultarle a Reynaldo esta información.         \item {\bfseries Walter Vargas (walter@covetel.com.ve)  } - {\bfseries 2011-07-19 14:42:49} \\ Saludos Roberto,  

En base a tu solicitud vamos a dejar las pruebas y continuar, esperando
entonces que sea consultada esta información al personal de CANTV.NET        \item {\bfseries Walter Vargas (walter@covetel.com.ve)  } - {\bfseries 2011-07-27 23:23:45} \\ Saludos,

El documento de el presente ticket indica en un parrafo textualmente lo
siguiente: 

 NOTA: Hasta el momento los únicos clientes de aplicaciones son courier-imap y
 los scripts de control que son utilizados vía línea de comando, sin embargo,
 se está planificando usar Sendmail con autenticación vía LDAP (smtp
 autenticado) para lo cual se adaptará el cliente “cyrus-sasl” el cual es
 utilizado por sendmail como vía de autenticación hacia LDAP.

Es importante saber si el servicio Sendmail ya esta utilizando autenticación
vía LDAP y cuales son los detalles de esta configuración.         \item {\bfseries Walter Vargas (walter@covetel.com.ve)  } - {\bfseries 2011-07-28 13:10:16} \\ A continuación, información suministrada por Reynaldo Martínez vía correo
eletrónico el día de hoy a las 8 de la mañana. 

 Saludos Walter, 
 
 Efectivamente ya sendmail está usando LDAP pero no solo vía cyrus-sasl. Estamos
 usando LDAP_ROUTING desde Sendmail para la validación de los mail-to/rcpt-to de
 cuentas @cantv.net y de dominios virtuales en el ISP (todos en LDAP) y también
 sendmail con cyrus-sasl para SMTP autenticado (SMTP-AUTH).
 
 También estamos usando LDAP para una validación primaria en la plataforma
 Webmail.
 
 Mas detalles no puedo darte sin autorización de Ender Mujica (lease,  los
 detalles exáctos de nuestra configuración) solo puedo decirte que ya estamos
 usando SENDMAIL con LDAP para las dos funciones mencionadas (ldap-routing y
 smtp-auth con cyrus-sasl).
 
 Atentamente, 
 
 Reynaldo R. Martinez P.
 Consultor de Operaciones TI Centralizadas
 Coordinación Soporte Middleware
 GGTO - GOC Gerencia Operaciones TI Centralizadas
 Telf.: +58 (212) 263-4526
 e-mail: rmarti05@cantv.com.ve        \item {\bfseries Walter Vargas (walter@covetel.com.ve)  } - {\bfseries 2011-07-28 13:17:27} \\ Saludos,  

Se requiere por favor solicitarle al Sr. Ender Mujica autorize al Sr. Reinaldo
Martínez suministrarnos información sobre los detalles técnicos de la
configuración de la plataforma de correo.

Este intercambio de información, puede realizarse a través de la plataforma de
gestión de casos. 

En dado caso de que sea requerido un usuario para Reinaldo Martínez, por favor
solicitarlo. 

Gracias.    \end{enumerate}

\end{description}

\section{Caso 92: Construir esquema Documento Análisis CANTV.NET  }

\begin{description}

\item[Asunto] Construir esquema Documento Análisis CANTV.NET \item[Descripción] Se debe construir el esquema para elaborar el documento de Análisis de
CANTV.NET\item[Propietario] Nobody\item[Horas Trabajadas] 0

\item[Comentarios] Lista de comentarios agregados a este caso:  
\begin{enumerate}
        \item {\bfseries  } - {\bfseries } \\         \item {\bfseries  } - {\bfseries } \\         \item {\bfseries  } - {\bfseries } \\         \item {\bfseries  } - {\bfseries } \\     \end{enumerate}

\end{description}

\section{Caso 91: Migrar Diagramas CANTV.NET a DIA }

\begin{description}

\item[Asunto] Migrar Diagramas CANTV.NET a DIA\item[Descripción] Se adjuntan los diagramas que deben ser migrados a DIA.\item[Propietario] cparedes\item[Horas Trabajadas] 16

\item[Comentarios] Lista de comentarios agregados a este caso:  
\begin{enumerate}
        \item {\bfseries Lilibeth Ramirez (lilibeth@covetel.com.ve)  } - {\bfseries 2011-07-12 18:45:21} \\ Migrando los diagramas del documento Visio-Flujo.Correos.pdf a formato libre
.DIA        \item {\bfseries Lilibeth Ramirez (lilibeth@covetel.com.ve)  } - {\bfseries 2011-07-13 17:56:42} \\ Se realizó una comparación de los archivos MAIL.Platform.Mayo.02.2011.pdf y
cantv_net.pdf y se determino que es la misma información en ambos documentos,
es por ello que se descarta MAIL.Platform.Mayo.02.2011.pdf y se renombre el
archivo cantv_net.pdf a cantv_net_plataforma_mayo_2011.pdf        \item {\bfseries Lilibeth Ramirez (lilibeth@covetel.com.ve)  } - {\bfseries 2011-07-13 19:42:47} \\ Se actualizo la información correspondiente a los documentos del repositorio
cantv-info, en donde fueron actualizados los nombres de algunos archivos y
fueron eliminados aquellos que se encontraban repetidos        \item {\bfseries Lilibeth Ramirez (lilibeth@covetel.com.ve)  } - {\bfseries 2011-07-13 23:12:25} \\ Se migro el diagrama de la página 1 del documento Visio-Flujo.Correos.pdf a Dia
y a PNG

Se adjunta los archivos a este ticket        \item {\bfseries Lilibeth Ramirez (lilibeth@covetel.com.ve)  } - {\bfseries 2011-07-14 03:24:52} \\ Se migro el diagrama de la página 2 del documento Visio-Flujo.Correo.pdf, se
adjuntan los diagramas en formato .dia y en formato .png        \item {\bfseries Lilibeth Ramirez (lilibeth@covetel.com.ve)  } - {\bfseries 2011-07-14 03:27:08} \\ Se migro el diagrama de la página 2 del documento Visio-Flujo.Correo.pdf, se
adjuntan los diagramas en formato .dia y en formato .png        \item {\bfseries Lilibeth Ramirez (lilibeth@covetel.com.ve)  } - {\bfseries 2011-07-14 03:28:31} \\ El diagrama que se encuentra en la página 4 del documento
Visio-Flujo.Correo.pdf, es identico al diagrama de la página 1, es por ello que
no se toma en cuenta ya que fue migrado anteriormente        \item {\bfseries Carlos Paredes (cparedes@covetel.com.ve)  } - {\bfseries 2011-07-14 03:39:15} \\ Migrado diagrama mail.cantv.net perteneciente a la pagina 1 del documento
cantv_net_plataforma_mayo_2011.pdf.

se adjunta diagrama        \item {\bfseries Carlos Paredes (cparedes@covetel.com.ve)  } - {\bfseries 2011-07-14 03:40:55} \\ Migrado diagrama relay.cantv.net perteneciente a la pagina 2 del documento
cantv_net_plataforma_mayo_2011.pdf.

se adjunta diagrama        \item {\bfseries Carlos Paredes (cparedes@covetel.com.ve)  } - {\bfseries 2011-07-14 03:41:47} \\ Migrado diagrama postman.ric2.cantv.net perteneciente a la pagina 3 del
documento cantv_net_plataforma_mayo_2011.pdf.

se adjunta diagrama        \item {\bfseries Carlos Paredes (cparedes@covetel.com.ve)  } - {\bfseries 2011-07-15 20:38:25} \\ Migrado diagrama dom-especial.ric2.cantv.net perteneciente a la pagina 4 del
documento cantv_net_plataforma_mayo_2011.pdf.

se adjunta diagrama        \item {\bfseries Carlos Paredes (cparedes@covetel.com.ve)  } - {\bfseries 2011-07-15 20:39:33} \\ Migrado diagrama pop-imap.cantv.net perteneciente a la pagina 5 del documento
cantv_net_plataforma_mayo_2011.pdf.

se adjunta diagrama        \item {\bfseries Carlos Paredes (cparedes@covetel.com.ve)  } - {\bfseries 2011-07-15 20:40:44} \\ Actualización de Tiempo en los 2 ultimos diagramas migrados
(dom-especial.ric2.cantv.net y pop-imap.cantv.net        \item {\bfseries Carlos Paredes (cparedes@covetel.com.ve)  } - {\bfseries 2011-07-15 20:41:33} \\ Migrado diagrama amavis-clamav perteneciente a la pagina 6 del documento
cantv_net_plataforma_mayo_2011.pdf.

se adjunta diagrama        \item {\bfseries Carlos Paredes (cparedes@covetel.com.ve)  } - {\bfseries 2011-07-15 20:42:31} \\ Migrado diagrama ldap-openldap perteneciente a la pagina 7 del documento
cantv_net_plataforma_mayo_2011.pdf.

se adjunta diagrama        \item {\bfseries Carlos Paredes (cparedes@covetel.com.ve)  } - {\bfseries 2011-07-15 20:43:53} \\ Migración de los diagramas en los archivos Visio-Flujo.Correos.pdf y
MAIL.Platform.Mayo.02.2011.pdf concluida.        \item {\bfseries Lilibeth Ramirez (lilibeth@covetel.com.ve)  } - {\bfseries 2011-07-17 20:04:53} \\ Fue Corregido el diagrama de la página 2 del documento Visio-Flujo.Correo.pdf,
se adjuntan los diagramas en formato .dia y en formato .png        \item {\bfseries Lilibeth Ramirez (lilibeth@covetel.com.ve)  } - {\bfseries 2011-07-17 20:10:19} \\ En la leyenda Color Amarillo de los Diagramas de la página 2 y 3 del Documento
Visio-Flujo.Correos.pdf, que son identicas a las que encontramos en el
documento INST-ODA-OAI-042-flujos-de-correo.pdf en la página 21 y 22 se
corrigio el siguiente texto:

Servicio Público balaceado

Y quedo corregido de la siguiente forma:

Servicio Público balanceado        \item {\bfseries Carlos Maldonado (cmaldonado@covetel.com.ve)  } - {\bfseries 2011-07-27 19:18:17} \\ Los diagramas: Flujo-Correo_02.dia y Flujo-Correo_01.dia deben ser reformados y
corregidos para que correspondan con el estado actual de la organización tal y
como se reseña en el attachment:

correo_reinaldo.txt

del ticket #98

https://rt.covetel.com.ve/Ticket/Display.html?id=98

CM    \end{enumerate}

\end{description}

\section{Caso 90: Crear Documento Análisis CANTV.NET }

\begin{description}

\item[Asunto] Crear Documento Análisis CANTV.NET\item[Descripción] Se requiere construir un documento de Análisis de la plataforma actual de
correo de CANTV.NET\item[Propietario] walter\item[Horas Trabajadas] 51

\item[Comentarios] Lista de comentarios agregados a este caso:  
\begin{enumerate}
        \item {\bfseries Lilibeth Ramirez (lilibeth@covetel.com.ve)  } - {\bfseries 2011-08-01 21:17:19} \\ Se debe corregir un error de transcripción en el documento de análisis doc_005,
especificamente en el Título: 2.1.7. F06, en:

1. Envío en formato SMS a RA.MOVILNET.COM.VE, de notificación de recepción de
correos en "cuantas" del dominio CANTV.NET que cuentan con el servicio TUN-TUN
(de correos recibidos por la capa RELAY.CANTV.NET).

Se debe cambiar " cuantas " por cuentas        \item {\bfseries Lilibeth Ramirez (lilibeth@covetel.com.ve)  } - {\bfseries 2011-08-01 21:28:42} \\ En le punto 2.1.13. F09. no se índica a quién fue generado el NDR:

6. NDR a generados por “tiempo excedido para la entrega” de correos enviados a
Domi- nios especiales (de correos recibidos por la capa RELAY.CANTV.NET y
MAIL.CANTV.NET).

Verificar si está o no incompleta ésta información.        \item {\bfseries Lilibeth Ramirez (lilibeth@covetel.com.ve)  } - {\bfseries 2011-08-01 21:42:00} \\ Ocurre lo mismo en los puntos:

2.1.11. F08
2.1.14. F16

El Lun Ago 01 17:28:42 2011, lilibeth escribió:
> En le punto 2.1.13. F09. no se índica a quién fue generado el NDR:
>
> 6. NDR a generados por “tiempo excedido para la entrega” de correos
> enviados a
> Domi- nios especiales (de correos recibidos por la capa
> RELAY.CANTV.NET y
> MAIL.CANTV.NET).
>
> Verificar si está o no incompleta ésta información.        \item {\bfseries Lilibeth Ramirez (lilibeth@covetel.com.ve)  } - {\bfseries 2011-08-01 21:43:38} \\ Se debe corregir en el punto 3.2.2. Hardware Utilizado: Recomendaciones ->
Disco duros: la palabra " recmoned " por recomendó.        \item {\bfseries Walter Vargas (walter@covetel.com.ve)  } - {\bfseries 2011-08-03 01:59:07} \\ Se requiere una descripción formal del término GOTIC-CCUN, por favor enviar en cuanto puedan la descripción formal de este término para agregarlo al glosario        \item {\bfseries Walter Vargas (walter@covetel.com.ve)  } - {\bfseries 2011-08-03 02:00:52} \\ Se requiere la información del servicio de almacenamiento al que se refieren en los documentos de CANTV.NET bajo el término FILER        \item {\bfseries Walter Vargas (walter@covetel.com.ve)  } - {\bfseries 2011-08-03 12:10:18} \\ Se requiere información formal y detallada sobre el término MGMT utilizado en CANTV.NET        \item {\bfseries Walter Vargas (walter@covetel.com.ve)  } - {\bfseries 2011-08-03 12:11:43} \\ Se requieren las especificaciones del appliance CISCO-ACE utilizado para balancear la carga de los servidores de la plataforma de correo CANTV.NET        \item {\bfseries Walter Vargas (walter@covetel.com.ve)  } - {\bfseries 2011-08-10 18:36:23} \\ Es posible conocer el tiempo preciso que le toma al programa /usr/local/bin/boss2ldif.pl en generar los archivos  en formato  LDIF  bajo el directorio /var/tmp ?         \item {\bfseries Walter Vargas (walter@covetel.com.ve)  } - {\bfseries 2011-08-10 21:29:56} \\ Gracias por la pronta respuesta Reynaldo, 

Yo calcule 15.5166666666667 minutos.

Saludos. 

El Mié Ago 10 14:56:01 2011, RMarti05@Cantv.com.ve escribió:
> Buenas tardes,
> 
> El script que regenera toda la BD de LDAP anota en un archivo los
> tiempos de inicio y finalización de cada paso importante de la
> reconstrucción. Hace poco migramos los servidores LDAP a equipos mas
> potentes y nos tocó reconstruir. Este es el segmento relativo a lo que
> preguntas:
> 
> Iniciado proceso de reconstruccion LDIF-BOSS. Tiempo de arranque: Wed
> Jul 13 15:59:57 VET 2011
> Finalizado proceso de reconstruccion LDIF-BOSS. Tiempo final: Wed Jul
> 13
> 16:15:28 VET 2011
> 
> Poco mas de 15 minutos para reconstruir los LDIF, pero este es SOLO un
> paso de todo el proceso de recontrucción y ese tiempo está sujeto a
> las
> condiciones de BOSS (la base de dato central) que no controlamos.
> Luego
> esos LDIF tienen que ser incluidos en el backend (berkeley db's). Eso
> tarda también. El proceso "completo" hasta dejar el LDAP completamente
> operativo tarda cerca de 30 minutos (eso incluye los 15 minutos de
> boss2ldif).
> 
> Atentamente,
> 
> Reynaldo R. Martinez P.
> Consultor de Operaciones TI Centralizadas
> Coordinación Soporte Middleware
> GGTO - GOC Gerencia Operaciones TI Centralizadas
> Telf.: +58 (212) 263-4526
> e-mail: rmarti05@cantv.com.ve
> 
> 
> On 08/10/2011 02:06 PM, Walter Vargas via RT wrote:
> > Es posible conocer el tiempo preciso que le toma al programa
> /usr/local/bin/boss2ldif.pl en generar los archivos  en formato  LDIF
> bajo el directorio /var/tmp ?
> >

        \item {\bfseries Carlos Maldonado (cmaldonado@covetel.com.ve)  } - {\bfseries 2011-08-11 22:25:09} \\ añadiendo tiempo trabajado    \end{enumerate}

\end{description}

\section{Caso 76: Analizar Documento: Configuracion del Servicio DNS corporativo de CANTV.doc }

\begin{description}

\item[Asunto] Analizar Documento: Configuracion del Servicio DNS corporativo de CANTV.doc\item[Descripción] El documento se encuentra en el repositorio cantv-info.git

https://rt.covetel.com.ve/Ticket/Attachment/760/508/Configuracion%20del%20Servicio%20DNS%20corporativo%20de%20CANTV.doc\item[Propietario] cparedes\item[Horas Trabajadas] 3

\item[Comentarios] Lista de comentarios agregados a este caso:  
\begin{enumerate}
        \item {\bfseries Carlos Paredes (cparedes@covetel.com.ve)  } - {\bfseries 2011-07-01 22:40:34} \\ Analisis del Documento de configuracion de configuración del servicio DNS        \item {\bfseries Carlos Paredes (cparedes@covetel.com.ve)  } - {\bfseries 2011-07-05 23:16:22} \\ Conclusión de análisis del documento Configuración del Servicio DNS corporativo
de CANTV donde se especifican los servidores pertenecientes al servicio DNS
alpha01, alpha02 alfha03, alpha04, dnscorp0a y dnscorp0b, su configuración y
sus sites.    \end{enumerate}

\end{description}

\section{Caso 74: Analizar Documento: CANTVOperaciones2005.doc }

\begin{description}

\item[Asunto] Analizar Documento: CANTVOperaciones2005.doc\item[Descripción] https://rt.covetel.com.ve/Ticket/Attachment/793/556/CANTVOperaciones2005.doc\item[Propietario] lilibeth\item[Horas Trabajadas] 3

\item[Comentarios] Lista de comentarios agregados a este caso:  
\begin{enumerate}
        \item {\bfseries  } - {\bfseries } \\         \item {\bfseries  } - {\bfseries } \\         \item {\bfseries  } - {\bfseries } \\         \item {\bfseries  } - {\bfseries } \\     \end{enumerate}

\end{description}

\section{Caso 73: Analizar Documento: Situación actual y análisis de espacio de la Plataforma de Mensajeria Corporativa.doc }

\begin{description}

\item[Asunto] Analizar Documento: Situación actual y análisis de espacio de la Plataforma de Mensajeria Corporativa.doc\item[Descripción] https://rt.covetel.com.ve/Ticket/Attachment/793/565/Situacio%5CxCC%5Cx81n%20actual%20y%20ana%5CxCC%5Cx81lisis%20de%20espacio%20de%20la%20Plataforma%20de%20Mensajeria%20Corporativa.doc\item[Propietario] lilibeth\item[Horas Trabajadas] 5

\item[Comentarios] Lista de comentarios agregados a este caso:  
\begin{enumerate}
        \item {\bfseries Lilibeth Ramirez (lilibeth@covetel.com.ve)  } - {\bfseries 2011-07-06 07:21:38} \\ Contiene información actualizada referente a:

Los Servidores
Los Buzones de correo
Listas de Distribución
Contactos
Usuarios con direcciones de correo externas

Además de tener información actualizada de la Distribución actual por cluster        \item {\bfseries Lilibeth Ramirez (lilibeth@covetel.com.ve)  } - {\bfseries 2011-07-06 07:26:42} \\ Se utilizó parte de la información correspondiente a los buzones y su
distribución. La más actualizada luego de ser comparada con los demás
documentos que contienen información acerca de estos puntos.    \end{enumerate}

\end{description}

\section{Caso 72: Analizar Documento: Matriz_de_interfaces_(proveedores).xls }

\begin{description}

\item[Asunto] Analizar Documento: Matriz_de_interfaces_(proveedores).xls\item[Descripción] https://rt.covetel.com.ve/Ticket/Attachment/793/562/Matriz_de_interfaces_%26%2340%3Bproveedores%26%2341%3B.xls\item[Propietario] lilibeth\item[Horas Trabajadas] 2

\item[Comentarios] Lista de comentarios agregados a este caso:  
\begin{enumerate}
        \item {\bfseries Carlos Paredes (cparedes@covetel.com.ve)  } - {\bfseries 2011-07-06 01:33:09} \\ En el documento suministrado la información no esta definida con exactitud, ya
que se presenta a manera de ejemplo. El personal del proyecto de migración de
CANTV debe definir y corroborar esta información.        \item {\bfseries Carlos Paredes (cparedes@covetel.com.ve)  } - {\bfseries 2011-07-06 01:35:58} \\ En el documento suministrado la información no esta definida con exactitud, ya
que se presenta a manera de ejemplo. El personal del proyecto de migración de
CANTV debe definir y corroborar esta información.        \item {\bfseries Lilibeth Ramirez (lilibeth@covetel.com.ve)  } - {\bfseries 2011-07-06 07:14:47} \\ Este documento no pudo ser utilizado para la elaboración del análisis del
servicio Exchange Server 2003, debido a que no se tiene conocimiento de si es
un ejemplo (el contenido del documento esta comentado como Ejemplo) o si es
información actual.    \end{enumerate}

\end{description}

\section{Caso 71: Analizar Documento: Exchange Topology.jpg  }

\begin{description}

\item[Asunto] Analizar Documento: Exchange Topology.jpg \item[Descripción] https://rt.covetel.com.ve/Ticket/Attachment/793/553/Exchange%20Topology.jpg\item[Propietario] lilibeth\item[Horas Trabajadas] 1

\item[Comentarios] Lista de comentarios agregados a este caso:  
\begin{enumerate}
        \item {\bfseries  } - {\bfseries } \\         \item {\bfseries  } - {\bfseries } \\         \item {\bfseries  } - {\bfseries } \\         \item {\bfseries  } - {\bfseries } \\     \end{enumerate}

\end{description}

\section{Caso 70: Analizar Documento: Exchange Topology 2006.jpg }

\begin{description}

\item[Asunto] Analizar Documento: Exchange Topology 2006.jpg\item[Descripción] https://rt.covetel.com.ve/Ticket/Attachment/793/554/Exchange%20Topology%202006.jpg\item[Propietario] lilibeth\item[Horas Trabajadas] 2

\item[Comentarios] Lista de comentarios agregados a este caso:  
\begin{enumerate}
        \item {\bfseries  } - {\bfseries } \\         \item {\bfseries  } - {\bfseries } \\         \item {\bfseries  } - {\bfseries } \\         \item {\bfseries  } - {\bfseries } \\     \end{enumerate}

\end{description}

\section{Caso 69: Analizar Documento: Configuración del Storage Exchnas.doc }

\begin{description}

\item[Asunto] Analizar Documento: Configuración del Storage Exchnas.doc\item[Descripción] https://rt.covetel.com.ve/Ticket/Attachment/793/558/Configuracio%5CxCC%5Cx81n%20del%20Storage%20Exchnas.doc\item[Propietario] lilibeth\item[Horas Trabajadas] 1

\item[Comentarios] Lista de comentarios agregados a este caso:  
\begin{enumerate}
        \item {\bfseries  } - {\bfseries } \\         \item {\bfseries  } - {\bfseries } \\         \item {\bfseries  } - {\bfseries } \\         \item {\bfseries  } - {\bfseries } \\     \end{enumerate}

\end{description}

\section{Caso 68: Analizar Documento: DescripcionGeneralDelSistemadeMensajeriadeCantv.doc }

\begin{description}

\item[Asunto] Analizar Documento: DescripcionGeneralDelSistemadeMensajeriadeCantv.doc\item[Descripción] El documento se encuentra en el repositorio cantv-info.git

https://rt.covetel.com.ve/Ticket/Attachment/793/559/DescripcionGeneralDelSistemadeMensajeriadeCantv.doc\item[Propietario] lilibeth\item[Horas Trabajadas] 11

\item[Comentarios] Lista de comentarios agregados a este caso:  
\begin{enumerate}
        \item {\bfseries Lilibeth Ramirez (lilibeth@covetel.com.ve)  } - {\bfseries 2011-07-01 22:47:29} \\ Durante el análisis del documento:
DescripcionGeneralDelSistemadeMensajeriadeCantv.doc se determino lo siguiente:

La información referente a Plataforma Exchange de CANTV, es coincidente con los
documentos CantvOperaciones 2007.doc y NAS_EXCHANGE.ppt. Además de coincidir
con el Diagrama Red Exchange.jpg y Exchange Topology.jpg

Presenta algunas variaciones con respecto al Esquema de almacenamiento de los
servidores Exchange que deben ser verificados por el personal de Cantv

Las rutas de registros de las transacciones en el Storage Group es la misma en
el Documento de CantvOperaciones 2007.doc

Las rutas de los archivos de Base de Datos es la misma en el Documento
CantvOperaciones 2007.doc

Los tipos de clientes coinciden en el documento solo se encuentra en este
documento, pero persiste la duda de si se encuentra o no actualizada esta
información

La Distribución de los Buzones de Correos por Storage Groupse se encuentra
desactualizada en este documento, por ello fue tomada del documento Situación
actual y análisis de espacio de la Plataforma de Mensajeria Corporativa.doc

El número de usuarios de Correo Cantv se encuentra desactualizado, por ello fue
tomada la información que se encuentra en el documento Situación actual y
análisis de espacio de la Plataforma de Mensajeria Corporativa.doc        \item {\bfseries Lilibeth Ramirez (lilibeth@covetel.com.ve)  } - {\bfseries 2011-07-06 07:34:50} \\ Este documento contenia un resumen general del servicio de Mensajeria Exchange
Server 2003, que fue analizado para formar parte de la documentación inicial de
éste servicio.    \end{enumerate}

\end{description}

\section{Caso 67: Analizar Documento: Guia de Operaciones Exchange.doc }

\begin{description}

\item[Asunto] Analizar Documento: Guia de Operaciones Exchange.doc\item[Descripción] El documento se encuentra en el repositorio cantv-info.git

https://rt.covetel.com.ve/Ticket/Attachment/793/560/Guia%20de%20Operaciones%20Exchange.doc\item[Propietario] lilibeth\item[Horas Trabajadas] 1

\item[Comentarios] Lista de comentarios agregados a este caso:  
\begin{enumerate}
        \item {\bfseries  } - {\bfseries } \\         \item {\bfseries  } - {\bfseries } \\         \item {\bfseries  } - {\bfseries } \\         \item {\bfseries  } - {\bfseries } \\     \end{enumerate}

\end{description}

\section{Caso 66: Analizar Documento: Red Exchange.jpg }

\begin{description}

\item[Asunto] Analizar Documento: Red Exchange.jpg\item[Descripción] El documento se encuentra en el repositorio cantv-info.git

https://rt.covetel.com.ve/Ticket/Attachment/793/555/Red%20Exchange.jpg\item[Propietario] lilibeth\item[Horas Trabajadas] 1

\item[Comentarios] Lista de comentarios agregados a este caso:  
\begin{enumerate}
        \item {\bfseries  } - {\bfseries } \\         \item {\bfseries  } - {\bfseries } \\         \item {\bfseries  } - {\bfseries } \\         \item {\bfseries  } - {\bfseries } \\     \end{enumerate}

\end{description}

\section{Caso 65: Analizar Documento: Manual de Operaciones - CANTV-11-2005.doc }

\begin{description}

\item[Asunto] Analizar Documento: Manual de Operaciones - CANTV-11-2005.doc\item[Descripción] El documento se encuentra en el repositorio cantv-info.git

https://rt.covetel.com.ve/Ticket/Attachment/793/561/Manual%20de%20Operaciones%20-%20CANTV-11-2005.doc\item[Propietario] lilibeth\item[Horas Trabajadas] 3

\item[Comentarios] Lista de comentarios agregados a este caso:  
\begin{enumerate}
        \item {\bfseries Walter Vargas (walter@covetel.com.ve)  } - {\bfseries 2011-07-01 19:58:54} \\ Este documento fue sometido a un proceso de DIFF contra el documento CANTV
Operaciones2007.doc, se analizo que información era útil para agregar al
documento principal.        \item {\bfseries Lilibeth Ramirez (lilibeth@covetel.com.ve)  } - {\bfseries 2011-07-06 07:30:31} \\ Fue analizado y solo fue utilizada la información importante que fue extraída
de la comparación hecha anteriormente.    \end{enumerate}

\end{description}

\section{Caso 64: Analizar Documento CANTV Operaciones2007.doc }

\begin{description}

\item[Asunto] Analizar Documento CANTV Operaciones2007.doc\item[Descripción] El documento se encuentra en el repositorio cantv-info.git

https://rt.covetel.com.ve/Ticket/Attachment/793/557/CANTV%20Operaciones2007.doc\item[Propietario] lilibeth\item[Horas Trabajadas] 35

\item[Comentarios] Lista de comentarios agregados a este caso:  
\begin{enumerate}
        \item {\bfseries Walter Vargas (walter@covetel.com.ve)  } - {\bfseries 2011-07-01 19:54:13} \\ Se realizo una comparación precisa con los otros dos documentos que contienen
información muy similar a este:

* CANTVOperaciones2005.doc
* Manual de Operaciones - CANTV-11-2005.doc        \item {\bfseries Lilibeth Ramirez (lilibeth@covetel.com.ve)  } - {\bfseries 2011-07-06 06:55:11} \\ En este documento se trabajo en detalle debido a que se comparo con 2
documentos más con el mismo contenido pero con fechas diferentes.

El contenido de este documento fue comparado con los archivos NAS_Exchange.ppt,
Situación actual y análisis de espacio de la Plataforma de Mensajeria
Corporativa.doc y DescripcionGeneralDelSistemadeMensajeriadeCantv.doc, ya que
contenian información similar, en algunos casos uno más actualizado que otro.

Se observo que el documento Cantv Operaciones2007.doc, contiene mucha
información acerca de la forma en la que debe ser operado el servicio Exchange
Server 2003, referenciando a la Guia de Operaciones de Exchange 2003.    \end{enumerate}

\end{description}

\section{Caso 63: Analizar Documento NAS_EXCHANGE.ppt }

\begin{description}

\item[Asunto] Analizar Documento NAS_EXCHANGE.ppt\item[Descripción] El documento se encuentra en el repositorio cantv-info

https://rt.covetel.com.ve/Ticket/Attachment/793/564/NAS_EXCHANGE.ppt\item[Propietario] lilibeth\item[Horas Trabajadas] 6

\item[Comentarios] Lista de comentarios agregados a este caso:  
\begin{enumerate}
        \item {\bfseries Walter Vargas (walter@covetel.com.ve)  } - {\bfseries 2011-07-01 19:45:52} \\ Este documento es una presentación en formato de Microsoft Power Point,
compuesto por 4 láminas que contienen la información de los volumenes lógicos
utilizados para conformar el almacenamiento de Exchange.

La información del tamaño de algunas LUNs entre las láminas 1, 2 y 3 no
concuerda con la información de la gráfica de la lámina 4.        \item {\bfseries Lilibeth Ramirez (lilibeth@covetel.com.ve)  } - {\bfseries 2011-07-01 23:29:57} \\ Este documento contiene la información más actualizada acerca de la
distribución de la NAS del Exchange.

Fue analizado y comparado con la información que se encuentra en el documento
DescripcionGeneralDelSistemadeMensajeriadeCantv.doc el cual contienen
información de la NAS pero se encuentra desactualizada.        \item {\bfseries Lilibeth Ramirez (lilibeth@covetel.com.ve)  } - {\bfseries 2011-07-06 07:05:31} \\ Este documento fue comparado con cada uno de los documentos que contenia
información acerca de la NAS del Exchange y se encontraba desactualizado.

Por ello no fue utilziado para elaborar el documento de analizis del Exchange
Server 2003    \end{enumerate}

\end{description}

\section{Caso 62: Analizar Documento Matriz funcional exchange (usuarios).xls }

\begin{description}

\item[Asunto] Analizar Documento Matriz funcional exchange (usuarios).xls\item[Descripción] El documento se encuentra en el repositorio cantv-info.git

https://rt.covetel.com.ve/Ticket/Attachment/793/563/Matriz%20funcional%20exchange%20%26%2340%3Busuarios%26%2341%3B.xls\item[Propietario] lilibeth\item[Horas Trabajadas] 4

\item[Comentarios] Lista de comentarios agregados a este caso:  
\begin{enumerate}
        \item {\bfseries Walter Vargas (walter@covetel.com.ve)  } - {\bfseries 2011-07-01 19:32:25} \\ Este documento es una Hoja de Calculo en Excel que contiene las características
de la plataforma de Correo actual Exchange, y estas características fueron
transformadas en requerimientos funcionales.

Es importante conocer si la gente de CANTV.NET o Middleware esta al tanto de
estos requerimientos.        \item {\bfseries Lilibeth Ramirez (lilibeth@covetel.com.ve)  } - {\bfseries 2011-07-06 07:10:22} \\ Esta información fue utilizada para el documento de analizis del servicio
Exchange Service 2003    \end{enumerate}

\end{description}

\section{Caso 61: Analizar Documento Matriz funcional (Wins).xls  }

\begin{description}

\item[Asunto] Analizar Documento Matriz funcional (Wins).xls \item[Descripción] El documento se encuentra en el repositorio cantv-info en la carpeta Directorio
Activo.

Tambien se encuentra en la direción:

https://rt.covetel.com.ve/Ticket/Attachment/767/517/Matriz%20funcional%20%26%2340%3BWins%26%2341%3B.xls\item[Propietario] cmaldonado\item[Horas Trabajadas] 1

\item[Comentarios] Lista de comentarios agregados a este caso:  
\begin{enumerate}
        \item {\bfseries Carlos Maldonado (cmaldonado@covetel.com.ve)  } - {\bfseries 2011-07-01 17:15:07} \\ El archivo solo contiene la descripción del servicio Wins y Netbios de la plataforma informática de CANTV y su obligatoriedad como funcionalidad migrada dentro de los entregables del proyecto        \item {\bfseries Carlos Maldonado (cmaldonado@covetel.com.ve)  } - {\bfseries 2011-07-06 21:38:50} \\ documento descriptivo sobre los requerimientos para migración de servicios NetBios y WINS    \end{enumerate}

\end{description}

\section{Caso 60: Analizar Documento: LDAP_Matriz funcional (usuarios).xls }

\begin{description}

\item[Asunto] Analizar Documento: LDAP_Matriz funcional (usuarios).xls\item[Descripción] El documento se encuentra en la carpeta Directorio Activo en el repositorio
cantv-info.git

Tambien se encuentra en la dirección:

https://rt.covetel.com.ve/Ticket/Attachment/767/517/Matriz%20funcional%20%26%2340%3BWins%26%2341%3B.xls\item[Propietario] cmaldonado\item[Horas Trabajadas] 1

\item[Comentarios] Lista de comentarios agregados a este caso:  
\begin{enumerate}
        \item {\bfseries Carlos Maldonado (cmaldonado@covetel.com.ve)  } - {\bfseries 2011-07-01 17:21:51} \\ el título de este ticket necesita ser cambiado a

LDAP_Matriz funcional (usuarios)

de otra forma sería un duplicado del ticket #61

intenté cambiar el título pero no fue posible

On Thu Jun 30 17:32:05 2011, walter wrote:
> El documento se encuentra en la carpeta Directorio Activo en el
> repositorio
> cantv-info.git
>
> Tambien se encuentra en la dirección:
>
>
https://rt.covetel.com.ve/Ticket/Attachment/767/517/Matriz%20funcional%20%26%2340%3BWins%26%2341%3B.xls        \item {\bfseries Walter Vargas (walter@covetel.com.ve)  } - {\bfseries 2011-07-01 23:21:21} \\ El documento para analizar se encuentra en el repositorio cantv-info bajo el
nombre LDAP_Matriz funcional (usuarios).xls

Tambien se encuentra adjunto a un ticket bajo la dirección:

https://rt.covetel.com.ve/Ticket/Attachment/767/518/LDAP_Matriz%20funcional%20%26%2340%3Busuarios%26%2341%3B.xls        \item {\bfseries Carlos Maldonado (cmaldonado@covetel.com.ve)  } - {\bfseries 2011-07-06 21:38:50} \\ documento con descripción sobre los requerimientos en términos de autentificación de scripts aplicaciones web y aplicaciones nativas, en cuanto a permisos de computadores y usuarios    \end{enumerate}

\end{description}

\section{Caso 59: Analizar Presentación Directorio_Activo_y_Manejo_de_Identidades.ppt }

\begin{description}

\item[Asunto] Analizar Presentación Directorio_Activo_y_Manejo_de_Identidades.ppt\item[Descripción] Este archivo se encuentra en el repositorio cantv-info.git bajo la carpeta
"Directorio Activo".

Tambien se encuentra adjunto al ticket 50 bajo la siguiente dirección:

https://rt.covetel.com.ve/Ticket/Attachment/767/516/Directorio_Activo_y_Manejo_de_Identidades.ppt\item[Propietario] cmaldonado\item[Horas Trabajadas] 2

\item[Comentarios] Lista de comentarios agregados a este caso:  
\begin{enumerate}
        \item {\bfseries  } - {\bfseries } \\         \item {\bfseries  } - {\bfseries } \\         \item {\bfseries  } - {\bfseries } \\         \item {\bfseries  } - {\bfseries } \\     \end{enumerate}

\end{description}

\section{Caso 58: Analizar Documento DirectorioActivoCANTV2003.ppt  }

\begin{description}

\item[Asunto] Analizar Documento DirectorioActivoCANTV2003.ppt \item[Descripción] \item[Propietario] cmaldonado\item[Horas Trabajadas] 3

\item[Comentarios] Lista de comentarios agregados a este caso:  
\begin{enumerate}
        \item {\bfseries Carlos Maldonado (cmaldonado@covetel.com.ve)  } - {\bfseries 2011-07-01 00:51:54} \\ la revisión de este documento requiere haber leído el Capítulo Dos de "Active Directory" de O'Reilly, llamado "
Active Directory Fundamentals"        \item {\bfseries Carlos Maldonado (cmaldonado@covetel.com.ve)  } - {\bfseries 2011-07-01 17:15:08} \\ el documento contiene la estructura general de los roles y servicios Active Directory instalados por servidor en CANTV

la terminología de roles empleada es errónea y el nuevo documento contendrá las correcciones pertinentes

información miscelánea sobre procedimientos para ver cuentas activas e inactivas en el dominio

descripción superficial de la entidad usuario en el esquema del Directorio Activo        \item {\bfseries Carlos Maldonado (cmaldonado@covetel.com.ve)  } - {\bfseries 2011-07-01 20:41:53} \\ Diapositiva 2

Alfha03 (typo)

¿Alguien puede verificar si este servidor en realidad tiene 3.6 GB de RAM?

No tiene sentido para mi, probablemente sea un error de tipeo, puede que tenga
3 o 6 GB de RAM, o 3.5GB, pero 3.6 es altamente improbable

CM

On Fri Jul 01 13:15:08 2011, cmaldonado wrote:
> el documento contiene la estructura general de los roles y servicios
> Active Directory instalados por servidor en CANTV
>
> la terminología
> de roles empleada es errónea y el nuevo documento contendrá las
> correcciones pertinentes
>
> información miscelánea sobre
> procedimientos para ver cuentas activas e inactivas en el dominio
> descripción superficial de la entidad usuario en el esquema del
> Directorio Activo        \item {\bfseries Carlos Maldonado (cmaldonado@covetel.com.ve)  } - {\bfseries 2011-07-06 22:30:28} \\ leído el archivo y analizada la información que debe ser transcrita al núevo documento de análisis en el ticket #57        \item {\bfseries Walter Vargas (walter@covetel.com.ve)  } - {\bfseries 2011-06-30 21:22:56} \\ Este documento se encuentra en el repositorio cantv-info.git, bajo la carpeta
Directorio Activo,
tambien se encuentra adjunto en el ticket #50 bajo el siguiente enlace:

https://rt.covetel.com.ve/Ticket/Attachment/767/514/DirectorioActivoCANTV2003.ppt    \end{enumerate}

\end{description}

\section{Caso 57: Crear Documento Análisis de Infraestructura de Directorio Activo doc_003 }

\begin{description}

\item[Asunto] Crear Documento Análisis de Infraestructura de Directorio Activo doc_003\item[Descripción] Título del documento: Infraestructura de Directorio Activo CANTV
Subtitulo: Análisis Preliminar de la Plataforma Actual
Identificador: doc_003
Repositorio: cantv-doc.git
Ruta: Documentos/doc_003/
Fuentes: Documentos/doc_003/src/

Documentos para Analizar:
* DirectorioActivoCANTV2003.ppt * Directorio_Activo_y_Manejo_de_Identidades.ppt
* Matriz funcional (Wins).xls * LDAP_Matriz funcional (usuarios).xls

Los documentos a analizar se encuentran en el repositorio cantv-info.git, en la
carpeta "Directorio Activo"\item[Propietario] cmaldonado\item[Horas Trabajadas] 16

\item[Comentarios] Lista de comentarios agregados a este caso:  
\begin{enumerate}
        \item {\bfseries Walter Vargas (walter@covetel.com.ve)  } - {\bfseries 2011-07-06 12:37:54} \\ Se adjunta el documento que contiene el conjunto de cambios hasta la fecha para
la elaboración de este documento.        \item {\bfseries Carlos Maldonado (cmaldonado@covetel.com.ve)  } - {\bfseries 2011-07-06 21:42:05} \\ Contentivo de la información analizada y extraída de la documentación del Directorio Activo proveída por CANTV        \item {\bfseries Carlos Maldonado (cmaldonado@covetel.com.ve)  } - {\bfseries 2011-07-06 22:11:45} \\ Versión preliminar terminada para su revisión

On Wed Jul 06 17:42:05 2011, cmaldonado wrote:
> Contentivo de la información analizada y extraída de la documentación
> del Directorio Activo proveída por CANTV        \item {\bfseries Carlos Maldonado (cmaldonado@covetel.com.ve)  } - {\bfseries 2011-07-06 22:32:49} \\ documento creado con toda la información proveída por CANTV, en referencia a estructura del Directorio Activo, Tareas Documentadas y matrices funcionales de los servicios a migrar en el área de red y usuarios        \item {\bfseries Walter Vargas (walter@covetel.com.ve)  } - {\bfseries 2011-06-30 21:17:30} \\ Los diagramas correspondientes al Directorio Activo fueron migrados al formato
DIA por Carlos Paredes y se encuentran en:

https://rt.covetel.com.ve/Ticket/Display.html?id=21#txn-423    \end{enumerate}

\end{description}

\section{Caso 55: Documento Análisis de Plataforma de Correo Corporativo Exchange doc_002 }

\begin{description}

\item[Asunto] Documento Análisis de Plataforma de Correo Corporativo Exchange doc_002\item[Descripción] Escribir un documento formal que contenga el resultado del análisis de la
documentación de la plataforma de correo corporativo Exchange 2003.

Identificador

doc_002

Título

Infraestructura de Correo Corporativo Exchange Server 2003

Sub Título

Análisis Preliminar de la Plataforma Actual\item[Propietario] lilibeth\item[Horas Trabajadas] 45

\item[Comentarios] Lista de comentarios agregados a este caso:  
\begin{enumerate}
        \item {\bfseries Walter Vargas (walter@covetel.com.ve)  } - {\bfseries 2011-07-01 19:20:16} \\ Se adjunta el informe de actividades con su respectivo diff para este
documento.        \item {\bfseries Lilibeth Ramirez (lilibeth@covetel.com.ve)  } - {\bfseries 2011-07-06 06:59:13} \\ El documento fue elaborado y se encuentra en el repositorio cantv-doc.

Adjunto el mismo al cierre de este ticket        \item {\bfseries Walter Vargas (walter@covetel.com.ve)  } - {\bfseries 2011-07-06 12:34:12} \\ Se adjunta el conjunto de cambios necesarios para la elaboración de este
documento.        \item {\bfseries Walter Vargas (walter@covetel.com.ve)  } - {\bfseries 2011-07-06 12:42:53} \\ Se adjunta el detalle de los ficheros involucrados en el conjunto de cambios.    \end{enumerate}

\end{description}

\section{Caso 52: Analizar Documentos de Exchange 2003 }

\begin{description}

\item[Asunto] Analizar Documentos de Exchange 2003\item[Descripción] Se requiere analizar los documentos respectivos al Exchange 2003 y
posteriormente
elaborar documento sobre el servicio.

-- 
T.S.U. Lilibeth J. Ramírez M.
Dpto. Administrativo
Documentación Técnica
COVETEL, R.S.
San Cristóbal . Estado Táchira
0416-5023756
sip:102@asterisk.covetel.com.ve <sip%3A103@asterisk.covetel.com.ve>\item[Propietario] lilibeth\item[Horas Trabajadas] 0

\item[Comentarios] Lista de comentarios agregados a este caso:  
\begin{enumerate}
        \item {\bfseries  } - {\bfseries } \\         \item {\bfseries  } - {\bfseries } \\         \item {\bfseries  } - {\bfseries } \\         \item {\bfseries  } - {\bfseries } \\     \end{enumerate}

\end{description}

\section{Caso 50: Analizar Documentos Directorio Activo }

\begin{description}

\item[Asunto] Analizar Documentos Directorio Activo\item[Descripción] Se requiere analizar los documentos respectivos sobre el Directorio Actico y
posteriormente elaborar documento\item[Propietario] cmaldonado\item[Horas Trabajadas] 2

\item[Comentarios] Lista de comentarios agregados a este caso:  
\begin{enumerate}
        \item {\bfseries Carlos Maldonado (cmaldonado@covetel.com.ve)  } - {\bfseries 2011-07-06 22:27:58} \\ el documento Diseno.y.Operacion.Solucion.OpenLDAP.pdf

no corresponde a los documentos que deben ser analizados

por lo tanto este documento debe ser removido de la lista de adjuntos

CM


On Wed Jun 29 20:21:41 2011, cparedes wrote:
<br />&gt; Se requiere analizar los documentos respectivos sobre el Directorio
Actico y
<br />&gt; posteriormente elaborar documento
<br />
<br />
<br />        \item {\bfseries Carlos Maldonado (cmaldonado@covetel.com.ve)  } - {\bfseries 2011-07-06 22:30:27} \\ documentos analizados, novedades reportadas en el ticket correspondiente a cada documento    \end{enumerate}

\end{description}

\section{Caso 49: Analizar Documentos sobre DNS }

\begin{description}

\item[Asunto] Analizar Documentos sobre DNS\item[Descripción] Se requiere analizar los documentos respectivos al DNS y posteriormente
elaborar documento sobre el servicio.\item[Propietario] cparedes\item[Horas Trabajadas] 6

\item[Comentarios] Lista de comentarios agregados a este caso:  
\begin{enumerate}
        \item {\bfseries  } - {\bfseries } \\         \item {\bfseries  } - {\bfseries } \\         \item {\bfseries  } - {\bfseries } \\         \item {\bfseries  } - {\bfseries } \\     \end{enumerate}

\end{description}

\section{Caso 47: Analizar Información relacionada al correo CANTV.NET }

\begin{description}

\item[Asunto] Analizar Información relacionada al correo CANTV.NET\item[Descripción] Se obtuvo la siguiente información via correo electrónico:

1.- Las capas postman-mail y postman-relay fueron unificadas en una
sola. Las capas mail y relay envían los correos con destino @cantv.net
(y cuentas virtuales) hacia postman (postman.ric2.cantv.net) y ahi son
entregados a los buzones.

2.- La capa postman no hace las consultas ya vía DNS-D (de hecho no
existe esa capa de dns-d para correo). Las consultas son ahora vía LDAP
hacia la capa "ldap.ric2.cantv.net".

3.- Las capas POP e IMAP fueron unificadas y ahora se llaman POP-IMAP.
Pop-imap consulta la capa LDAP basada en OpenLDAP (ldap.ric2.cantv.net)
para todos los permisos de usuario.

4.- No aparece en estos flujos la capa AMAVIS/CLAMAV (la capa
AntiSPAM/AntiVIRUS). Las capas MAIL y RELAY validan cada correo entrante
por varios métodos (dnsbl, milters para SPF, controles adicionales vía
LDAP hacia ldap.ric2.cantv.net y finalmente por virus y contenido hacia
la capa AntiSPAM amavis/clamav "amavis.ric2.cantv.net").

5.- XMAIL (la capa front-end de correo web) fue sustituida por NITIDO-WEB.

6.- La capa LDAP fue sustituida por OpenLDAP (ldap.ric2.cantv.net).

7.- El balanceador actual es el CISCO ACE y todas las capas están
balanceadas ahi !.

8.- La comunicación entre capas se hace por la red RIC2, la comunicación
entre POP-IMAP y POSTMAN hacia los filers de buzones se hace por la red
RAL2. Las labores de monitoreo y consultas DNS se hacen por la res de
administración "MGMT" y finalmente el servicio se da por la red pública
de servicio "SVC".

9.- Estamos dando servicios encriptados para todas las capas front-end.
Específicamente:

MAIL: vía tls sobre puerto 25 y ssl sobre puerto 465.
RELAY: vía tls sobre puerto 25.
POP: vía tls sobre puerto 110 y ssl sobre 995.

El servicio IMAP solo se da internamente (a la red corporativa de cantv).
El servicio MAIL está abierto solamente (puertos 25 y 465) a las redes
de cantv (acceso, corporativos, dedicados, etc). Para el resto del
planeta está cerrada (vía iptables en cada servidor).
El servicio POP está abierto a todo el mundo.
El servicio de correo web está abierto a todo el mundo.

Junto con los documentos viejos (en revisión) te anexo el diagrama
actualizado de la plataforma de correo del ISP
(mail.platform.mayo.02.2011.pdf). Ese si es el que mantenemos
religiosamente actualizado y tiene todo el mapa actual de la plataforma
de correo del ISP y su relación entre capas.

Recuerda que los documentos de flujos están en "revisión" y el único
documento actualizado el el diagrama de la plataforma de correo.

Atentamente,

Reynaldo R. Martinez P.
Consultor de Operaciones TI Centralizadas
GGTO - GOC Gerencia Operaciones TI Centralizadas
Coordinación Soporte Middleware (antigua gotic-ccun)
Telf.: +58 (212) 263-4526
e-mail: rmarti05@cantv.com.ve\item[Propietario] walter\item[Horas Trabajadas] 0

\item[Comentarios] Lista de comentarios agregados a este caso:  
\begin{enumerate}
        \item {\bfseries Lilibeth Ramirez (lilibeth@covetel.com.ve)  } - {\bfseries 2011-07-06 07:38:23} \\ Esta actividad fue culminada con éxito        \item {\bfseries Walter Vargas (walter@covetel.com.ve)  } - {\bfseries 2011-07-07 07:53:50} \\ Hubo un error al cerrar este ticket, aun falta analizar el material de
CANTV.NET    \end{enumerate}

\end{description}

\section{Caso 42: Reunión 30 Junio 2011 Servicio DHCP }

\begin{description}

\item[Asunto] Reunión 30 Junio 2011 Servicio DHCP\item[Descripción] Título: Mesa Trabajo (procesos, técnica) Servicio DHCP

Dirección: Los Cortijos 2 Piso 3 Sala 2 (Sala pequeña)

Cuando: Jue 30 Jun 2011 08:00 – 11:45

Organizador: jblond01@cantv.com.ve

Descripción:

Mesa para levantamiento de información tecnica y de procesos asociados 
al servicio DHCP.
  Agenda:
  8:00 am - 9:00 am Levantamiento de Procesos Asociados
  9:00 am - 11:45 am Levantamiento de Procesos Técnicos

Saludos Cordiales

-- 

Jorge A Blondell C
Especialista de Proyectos
Proyecto Migración Escritorio Corporativo a Software Libre
Gerencia de Programa Soluciones de TI
Gerencia General de Proyectos Mayores
Centro de Negocios Cantv Los Cortijos, Edificio 2, Piso 3
Movil : 58 - 416 - 6185295
Ofic:   58 - 212 - 5005286
\item[Propietario] juan\item[Horas Trabajadas] 2

\item[Comentarios] Lista de comentarios agregados a este caso:  
\begin{enumerate}
        \item {\bfseries Walter Vargas (walter@covetel.com.ve)  } - {\bfseries 2011-07-01 20:20:50} \\ La reunión se realizo satisfactoriamente, el recurso asignado por Covetel Juan
Mesa estuvo a las 8:15 en las instalaciones y salió a las 10:15.

Participantes:
* Lidia Briceño, GGPM/GO Proyecto, lbrice@cantv.com.ve, 02125000935
* Dino Gaspero, GOTID/RedCorpo, dgaspero@cantv.com.ve, 02125005670
* Franklin Araujo, COR/PAP, farauj01@cantv.com.ve, 02125006870
* Jorge Blondell, GOPM, jblonde01@cantv.com.ve, 02125005286
* Roberto de Oliveira, GGPM, rdeoli01@cantv.com.ve
* Juan Mesa, Covetel, juan@covetel.com.ve        \item {\bfseries Walter Vargas (walter@covetel.com.ve)  } - {\bfseries 2011-07-07 07:41:24} \\ Detalles de la reunion enviados por Juan Mesa:

Motivo de la reunión: El motivo principal de la reunión era la revisión de como
se lleva el manejo del DHCP en la organización. Comentarios: * Se hace una
breve explicación de como se maneja el servicio DHCP. * Se indica que la
gerencia encargada de llevar el DHCP solo usa las funciones básicas del DHCP de
Windows incluso no manejan características "avanzadas" del DHCP como la
creación o asignación de nuevas opciones de DHCP. * La unidad encargada del
manejo del DHCP indica que desea una interfaz de tipo GUI para el manejo del
nuevo DHCP, no quieren manipular archivos de configuración. * Existen 2
servidores en cluster para el manejo del DHCP de toda la organización, los
mismos están replicados, no se tiene información sobre como está configurado el
cluster Activo/Pasivo o Activo/Activo. * Los logs del servidor DHCP son
llevados por hosting windows. * Solo en casos especiales se hacen
configuraciones "avanzadas" sobre los tiempos en las leases del DHCP para
algunos scope. * No existen perfiles de usuario para el manejo del DHCP, los
encargados del manejo del DHCP se conectan al mismo vía escritorio remoto y
tienen acceso total a las configuraciones del servicio. * No se logra hacer una
demostración debido a la falta de conectividad en la sala de reunión, pero se
indica que no hay ningún manejo adicional a lo que brinda el DHCP de Windows,
Juan Mesa de COVETEL indica que sabe como se configura y administra un servidor
DHCP en Windows. * Posterior a la reunión Juan Mesa, Roberto De Oliveira y
Jorge Blondell sostienen una pequeña conversación y se llega a la conclusión de
que existe la posibilidad de que esta información y requerimientos no sean de
mayor importancia debido que "Middleware" podría llevar este servicio y ellos
lo llevaran a su manera, osea, editando los archivos de configuración, quizás
el único cambio será la adición de CFEngine para el manejo del servicio.    \end{enumerate}

\end{description}

\section{Caso 24: Reunión: Piso 5 Edificio NEA, Sala de Auditoria, Ala Norte }

\begin{description}

\item[Asunto] Reunión: Piso 5 Edificio NEA, Sala de Auditoria, Ala Norte\item[Descripción] Inicio de actividades relacionada a la migración de los servicios de apoyo al
escritorio corporativo (DNS, DHCP, Correo y Colaboración, Directorio,
autenticación y impresión). La reunión tiene como objetivo realizar una
presentación del alcance del proyecto y establecer participantes, responsables
con la unidades encargadas de la continuidad operativa de estos servicios para
el inicio del PAP de la fase 2 del proyecto Migración del Escritorio
Corporativo a Software Libre.\item[Propietario] juan\item[Horas Trabajadas] 2

\item[Comentarios] Lista de comentarios agregados a este caso:  
\begin{enumerate}
        \item {\bfseries  } - {\bfseries } \\         \item {\bfseries  } - {\bfseries } \\         \item {\bfseries  } - {\bfseries } \\         \item {\bfseries  } - {\bfseries } \\     \end{enumerate}

\end{description}

\section{Caso 21: Migrar los diagramas construidos en formato de Microsoft Visio a DIA }

\begin{description}

\item[Asunto] Migrar los diagramas construidos en formato de Microsoft Visio a DIA\item[Descripción] Se requiere la migración de los siguientes archivos en formato de Microsoft
Visio a DIA:

   - AD OUs-Draw entire Active Directory Structure.vsd
   - Directorio Activo.vsd

Se adjuntan ambos archivos para su respectiva conversión

-- 
T.S.U. Lilibeth J. Ramírez M.
Dpto. Administrativo
Documentación Técnica
COVETEL, R.S.
San Cristóbal . Estado Táchira
0416-5023756
sip:102@asterisk.covetel.com.ve <sip%3A103@asterisk.covetel.com.ve>\item[Propietario] cparedes\item[Horas Trabajadas] 5

\item[Comentarios] Lista de comentarios agregados a este caso:  
\begin{enumerate}
        \item {\bfseries Carlos Paredes (cparedes@covetel.com.ve)  } - {\bfseries 2011-07-01 02:50:47} \\ Correccion de Diagrama de Directorio Activo        \item {\bfseries Carlos Maldonado (cmaldonado@covetel.com.ve)  } - {\bfseries 2011-07-06 22:14:13} \\ Publicando nuevamente el png con más tamaño para evitar su distorsión ilegible
en el documento

corrección de unas comillas dobles por typo de origen, donde debe ir un número
dos

corrección de la palabra Domain en vez de Domaind, typo en el documento
original de CANTV

correcciones realizadas en el archivo fuente .dia y adjuntado

CM

On Mon Jun 13 21:50:55 2011, lilibeth wrote:
> Se requiere la migración de los siguientes archivos en formato de Microsoft
> Visio a DIA:
>
> - AD OUs-Draw entire Active Directory Structure.vsd
> - Directorio Activo.vsd
>
> Se adjuntan ambos archivos para su respectiva conversión
>        \item {\bfseries Carlos Paredes (cparedes@covetel.com.ve)  } - {\bfseries 2011-06-15 19:00:56} \\ Conversión de .vsd a .png        \item {\bfseries Carlos Paredes (cparedes@covetel.com.ve)  } - {\bfseries 2011-06-17 16:35:33} \\ Adjunto archicos . dia y .png de los diagramas terminados    \end{enumerate}

\end{description}
